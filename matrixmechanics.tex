\documentclass{article}
\usepackage{amsmath}
\usepackage{bm}
\usepackage{enumitem}
\usepackage{fancyhdr}
\usepackage[a4paper,total={6.5in,8.5in}]{geometry}
% \usepackage[a4paper,total={6.5in,9in}]{geometry}
\usepackage{physics}
\begin{document}
\pagestyle{fancy}
\fancyhf{}
\rhead{Name:\ \underline{\hspace{6cm}}}
\lhead{Math activity 2: matrix mechanics}
\rfoot{\thepage}
\renewcommand{\headrulewidth}{0pt}
\noindent *In this worksheet, assume all vectors are normalized, i.e., $\mathbf{v}^\dagger\mathbf{v} = 1$
\section{Hermitian and unitary matrices}
  \begin{enumerate}[label=(\alph*)]
      \item Show that the eigenvalues of a Hermitian matrix are real.
      \vspace{3cm}
      \item Using the fact that the eigenvalues of a Hermitian matrix $H$ are real, show that if the eigenvalues are nondegenerate, the corresponding eigenvectors are orthogonal, i.e., show that
      \begin{equation}
      \mathbf{v}^\dagger_m\mathbf{v}_n = \delta_{mn}
      \qquad\text{if}\qquad
      H\mathbf{v}_n = \epsilon_n\mathbf{v}_n,
      \text{  and  }
      \epsilon_m \neq\epsilon_n
      \text{  for  }
      m\neq n
      \end{equation}
      \vspace{4cm}
      %
      where we are assuming that each $\mathbf{v}_n$ is normalized.
      \item Let $U$ be a square matrix whose columns are the normalized eigenvectors of a Hermitian matrix, and let $I$ be the identity matrix so that
      \begin{equation}
          I_{mn} = \delta_{mn}
          \qquad\text{and}\qquad
          I\mathbf{v} = \mathbf{v}
      \end{equation}
      %
      for any vector $\mathbf{v}$. Show that
      \begin{equation}
          \label{eq:unitary}
          U^\dagger U = UU^\dagger = I.
      \end{equation}
      \vspace{2cm}
      %
      Note: by definition, all unitary operators obey equation \ref{eq:unitary}!
  \end{enumerate}
\newpage
\section{Functions of matrices}
  \begin{enumerate}[label=(\alph*)]
      \item\label{it:D} Consider the diagonal matrix
      \begin{equation}
      D = \mqty(a&0\\0&b),
      \end{equation}
      %
      where $a$ and $b$ are real numbers.  What is $D^n$?
      \vspace{1.5cm}
      \item What is $f(D)$ in terms of $f(a)$ and $f(b)$?
      \vspace{2cm}
      \item Consider a 2$\times$2 Hermitian matrix $A$ with eigenvalues $a_1$ and $a_2$. Let $U$ be a $2\times2$ matrix whose columns are the normalized eigenvectors of $A$.  Show that the diagonal matrix
      \begin{equation}
      D = \mqty(a_1&0\\0&a_2),
      \end{equation}
      %
      can be written as some product 
      $A$ and $U$.
      \vspace{2cm}
      \item Use equation \ref{eq:unitary} to write is $A$ in terms of $D$ and $U$.
      \vspace{2cm}
      \item What is $A^n$ in terms of $D$ and $U$? Use equation \ref{eq:unitary} to simplify the expression as much as possible.
      \vspace{2cm}
      \item\label{it:f(A)} Show that the function $f(A)$ can be written as
      \begin{equation}
      f(A)
      = U
      \mqty(f(a_1) &0\\0 &f(a_2))
      U^\dagger
      \end{equation}
      \vspace{2.5cm}
    \end{enumerate}
\newpage
\section{Application}
\noindent
The goal of this section is to find the matrix representation of the operator
$\hat{R}(\theta, \vb{n}) = e^{-i\theta\hat{\vb{J}}\cdot\vb{n}/\hbar}$, which rotates an arbitrary spin-1/2 state by an angle $\theta$ about an axis $\vb{n} = (n_x, n_y, n_z)$, letting $|\vb{n}|^2 = 1$.  Here, the components of $\hat{\vb{J}} = (\hat{J}_x, \hat{J}_y, \hat{J}_z)$ are the total angular momentum operators in the $x$, $y$, and $z$ directions. Their matrix representations in the $\ket{\pm z}$ basis are given in the text as
\begin{equation}
J_x = \frac{\hbar}{2}\mqty(0&1\\1&0),
\hspace{3cm}
J_y = \frac{\hbar}{2}\mqty(0&-i\\i&0),
\hspace{3cm}
J_z = \frac{\hbar}{2}\mqty(1&0\\0&-1).
\end{equation}
%
    \begin{enumerate}[label=(\alph*)]
      \item We will now apply the methods we worked out in parts 2\ref{it:D}-2\ref{it:f(A)}.  Write the $2\times2$ matrix representation of the operator $\hat{\vb{J}}\cdot\vb{n}$ in the $\ket{\pm z}$ basis in terms of $n_x$, $n_y$, and $n_z$. We'll call this matrix $\vb{J}\cdot\vb{n}$ (removing the ``hat'').
      \vspace{2.5cm}
      \item Find the eigenvalues and normalized eigenvectors of $\vb{J}\cdot\vb{n}$.
      \vspace{2.5cm}
      \item Write $\vb{J}\cdot\vb{n}$ as the product of a diagonal matrix and two unitary matrices.
      \vspace{2.5cm}
      \item Write $R(\theta, \vb{n}) = e^{-i\theta\vb{J}\cdot\vb{n}/\hbar}$, the matrix repesentation of $\hat{R}(\theta, \vb{n})$, as the product of a diagonal matrix and two unitary matrices.
      \vspace{2.5cm}
      \item Carry out the matrix product you arrived at in the previous question. Show that the resulting matrix can be written as
      \begin{equation}
      R(\theta,\vb{n})
      =
      I\cos\left(\frac{\theta}{2}\right)
      -
      \frac{2i}{\hbar}(\vb{J}\cdot\vb{n})\sin\left(\frac{\theta}{2}\right),
      \end{equation}
      %
      where $I$ is the identity matrix.
      \vspace{2.5cm}
    \end{enumerate}
\newpage
\section{Reflection}
    \begin{enumerate}[label=(\alph*)]
      \item Check that the matrix $R(\theta,\vb{n})$ correctly rotates the $\ket{+z}$ state into $\ket{-z}$, $\ket{+x}$, and $\ket{-x}$.
      \vspace{6cm}
      \item In a bulleted list, summarize all of the important steps that allowed us to write the matrix representation of the function of an operator.
  \end{enumerate}
    \newpage
 \section{Outer Products}
    
    \noindent
      Since $\ket{+z}$ and $\ket{-z}$ form a complete basis, any matrix can be formed out of a linear combination of direct products of the basis bras and kets.\newline
      
      Recall $\ket{+z}$ and $\ket{-z}$ may be represented in the $S_{z}$ basis as 
$\left(\begin{array}{c}1 & 0 \end{array}\right)$, and $ \left(\begin{array}{c}0 & 1 \end{array}\right)$ respectively.  It can also be seen that the outer product $\ket{+z}\bra{+z}$ may be represented as 
\begin{equation}
\left(\begin{array}{c}1 & 0 \end{array}\right)\left(\begin{array}{cccc}1 & 0  \end{array}\right) = \left(\begin{array}{cccc}1 & 0  \\ 0 & 0 \end{array}\right)
\end{equation}

and the outer product $\ket{+z}\bra{-z}$ may be represented as 

\begin{equation}
\left(\begin{array}{c}1 & 0 \end{array}\right)\left(\begin{array}{cccc}0 & 1  \end{array}\right) = \left(\begin{array}{cccc}0 & 1  \\ 0 & 0 \end{array}\right)
\end{equation}


      \begin{enumerate}[label=(\alph*)]
      \item Using outer products, find a way to form the Identity matrix out of outer products of the $\ket{+z}$ and $\ket{-z}$ bras and kets in the $S_{z}$ basis.
      \vspace{4cm}
      \item Similarly, form the Pauli matrix $\sigma_{z}.$
      \vspace{2cm}
      \item If you are up for a little challenge, form the Pauli $\sigma_{y}$, and $\sigma_{x}$ matrices out of a linear combination of outer products of $\ket{+z}$ and $\ket{-z}$ bras and kets in the $S_{z}$ basis.
  \end{enumerate}
\end{document}
